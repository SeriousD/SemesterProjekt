\subsection{Dreiecksgenerator}
\begin{figure}[H]
\centering
 \includegraphics[scale=0.55]{gfx/triangle_generator.pdf}
 \caption{Miller Integrator}
	\label{triangle} 
\end{figure}
Für die Generierung eines Dreiecksignals wurde ein Miller-Integrator verwendet.
Die Grundlage für die Schaltung liegt in einer Applicationnote von TI\footnote{ Texas Instruments,Appnote 20, Seite 24, www.ti.com/lit/an/snoa621c/snoa621c.pdf}
Das Grundprinzip dieser Schaltung besteht aus einem Intergrator, welcher auf einen invertierenden Schmitt-Trigger rückgekoppelt ist.
Der linke Operationsverstärker bildet mit $R4$ und $R3$ den Schmitt-Trigger. Der Integrator besteht aus $R8$,$C1$ und dem rechten Operationsverstärker in Abbildung \ref{triangle}.
Frequenzbestimmend sind in dieser Schaltung alle Bauteile. Durch $R4$ und $R3$ werden die Schaltschwellen, und damit die Amplitude bestimmt. Diese berechnet sich wie folgt: 
\begin{equation}
\frac{R_4}{R_3}\cdot V_{cc}
\end{equation}
Mit $R_8$ und $C_1$ wird die Flankensteilheit($k$) in $\frac{U}{t}$ bestimmt:
\begin{equation}
k=\frac{R_8 \cdot V_{cc}}{C_1}
\end{equation}
$C_1$ wurde mit $33nF$ fest gewählt. $U_{ein}$ ist die Betriebspannung($V_{cc}$), welche mit $9V$ gewählt wurde.